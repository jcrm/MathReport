\subsection{Code:}
In figure \ref*{sc:SourceCode} there is a snippet of the code used to create the application.
The code shows the process described earlier in which the user presses the left mouse button to start the movement of the object.
Parts of the code have been removed because it is not relevant to the motion of the object.
These lines of code follow and comments which start with /* and ends with */. 

\begin{figure}[H]
	\centering
	\begin{lstlisting}
	if (Input.GetMouseButton (0) && isPressed == false){
	 //Cast a Ray forwards and check to see if it hits the object.
	 Ray ray = Camera.main.ScreenPointToRay (Input.mousePosition);
	 RaycastHit hit;
	  //If the ray hits and object initalise the variables.
	  if (Physics.Raycast (ray, out hit, distance)){
		/*Start the timer and disable pressing the mouse button again.*/
		//Calculate the force * direction of force.
		Vector3 forceVector = (force* ray.direction);
		//Add force to force at a particular location on object. Use Impulse so its an instantaneous force.
		rigidbody.AddForceAtPosition(forceVector, hit.point, ForceMode.Impulse);
		/*Collect direction of force and the position the force hits.*/
		/*Output a screenshot.*/
	  }
	}
	\end{lstlisting}
	\caption{Code Snippet}
	\label{sc:SourceCode}
\end{figure}

The relevant part of the code starts where the force vector is calculated.
The force vector is calculated by times the force variable by the vector containing the direction of the ray that has been cast from the camera to the object.
This vector is always normalized so this calculation doesn't need to be performed.
This force vector is then applied to the rigid body at the position the user is looking at using a function provided by Unity3D.
This function also takes a third variable which tells the function that the type of force added is an impulse force and should happen instantaneously.