\subsection{Inertia Tensor:}
The inertia tensor, which is sometimes called the mass matrix, is a 3x3 matrix which hold the characteristics of a rigid body.
This including the moment of inertia for each of its axes stored along the diagonal of the matrix and the products of inertia \citep{millington2007game}.
\citet{millington2007game} describes the products of inertia as the “tendency of an object to rotate in a direction in which torque is being applied.”
An example of the phenomena is given as a child’s top that starts spinning in one direction but will then suddenly jump upside and spin in another direction \citep{millington2007game}. 
Below equation 2.6 is the inertia tensor matrix with the previously mention diagonal moments of inertia and the remaining products of inertia.
$A$, $B$, $C$ or equations 2.7 are the moments of inertia, while $F$, $G$, $H$ are the products of inertia or equations 2.8. 
\begin{equation}\label{ma:Inertia Tensor}
\begin{bmatrix}
  A & -H & -G \\
  -H & B & -F \\
  -G & -F & C
\end{bmatrix}
\end{equation}
\begin{equation}\label{eq:Moments of Inertia}
A=\sum\rho\left({y}^{2}+{z}^{2}\right)
\qquad
B=\sum\rho\left({x}^{2}+{z}^{2}\right)
\qquad
C=\sum\rho\left({x}^{2}+{y}^{2}\right)
\end{equation}
\begin{equation}\label{eq:Product of Inertia}
F=\sum\rho\left(yz\right)
\qquad
G=\sum\rho\left(xz\right)
\qquad
H=\sum\rho\left(xy\right)
\end{equation}
Equation 2.9-2.11 show the triple integrals needed to calculate the inertia tensor for any object that has uniform density. 
\begin{equation}\label{eq:AB Triple Integral}
A=\iiint_{V} \rho({y}^{2}+{z}^{2})\  dx dy dz
\qquad
B=\iiint_{V} \rho({x}^{2}+{z}^{2})\ dx dy dz
\end{equation}
\begin{equation}\label{eq:CF Triple Integral}
C=\iiint_{V} \rho({x}^{2}+{y}^{2})\ dx dy dz
\qquad
F=\iiint_{V} \rho(yz)\ dx dy dz
\end{equation}
\begin{equation}\label{eq:GH Triple Integral}
G=\iiint_{V} \rho(xz)\ dx dy dz
\qquad
H=\iiint_{V} \rho(xy)\ dx dy dz
\end{equation}