\subsection{Inertia Tensor:}\label{sc:IT}
The inertia tensor, which is sometimes called the mass matrix, is a 3x3 matrix which hold the characteristics of a rigid body.
This includes the moment of inertia for each of its axes which stored along the diagonal of the matrix and the products of inertia which are the remaining values \citep{millington2007game}.
\citet{millington2007game} describes the products of inertia as the “tendency of an object to rotate in a direction
different from the direction in which the torque is being applied.”
An example of the phenomena is given as a child's top that starts spinning in one direction but will then suddenly jump upside down and spin in another direction \citep{millington2007game}. 

Equation \ref{ma:Inertia Tensor} is the inertia tensor matrix with the previously mention diagonal moments of inertia and the remaining values representing the products of inertia.
$A$, $B$, $C$ or equation \ref{eq:Moments of Inertia} are the moments of inertia, while $F$, $G$, $H$ are the products of inertia or equation \ref{eq:Product of Inertia}.
In these equations $x$ is the distance from the centre of mass to the particle along the x-axis, similarly $y$ and $z$ are the distance along the y-axis and the z-axis respectively, while $m$ is the mass of the particle.
\begin{equation}\label{ma:Inertia Tensor}
	\begin{bmatrix}
	  A & -H & -G \\
	  -H & B & -F \\
	  -G & -F & C
	\end{bmatrix}
\end{equation}
\begin{equation}\label{eq:Moments of Inertia}
	A=\sum m\left({y}^{2}+{z}^{2}\right)
	\qquad
	B=\sum m\left({x}^{2}+{z}^{2}\right)
	\qquad
	C=\sum m\left({x}^{2}+{y}^{2}\right)
\end{equation}
\begin{equation}\label{eq:Product of Inertia}
	F=\sum m\left(yz\right)
	\qquad
	G=\sum m\left(xz\right)
	\qquad
	H=\sum m\left(xy\right)
\end{equation}

Equation \ref{eq:AB Triple Integral} - \ref{eq:GH Triple Integral} show the triple integrals needed to calculate the inertia tensor for any object that has uniform density.
In these equations like in equation \ref{eq:Moments of Inertia} and equation \ref{eq:Product of Inertia} $x$, $y$, and $z$ represent distance from the centre of mass along the corresponding axes.
$\rho$ is the density of the object and when equation \ref{eq:MassIntegral} is used from section \ref{sc:COM} it can be substituted later for the mass of the entire object.
\begin{equation}\label{eq:AB Triple Integral}
	A=\iiint \rho({y}^{2}+{z}^{2})\  dx dy dz
	\qquad
	B=\iiint \rho({x}^{2}+{z}^{2})\ dx dy dz
\end{equation}
\begin{equation}\label{eq:CF Triple Integral}
	C=\iiint \rho({x}^{2}+{y}^{2})\ dx dy dz
	\qquad
	F=\iiint \rho(yz)\ dx dy dz
\end{equation}
\begin{equation}\label{eq:GH Triple Integral}
	G=\iiint \rho(xz)\ dx dy dz
	\qquad
	H=\iiint \rho(xy)\ dx dy dz
\end{equation}

The inertia tensor and angular velocity are used in The Euler Equations to calculate the amount of torque needed to have an object rotate about a pivoted part of itself \citep{mactaggart2013l8}.
The equations to achieve this are equations \ref{eq:Gamma1} - \ref{eq:Gamma3} with $A$, $B$ and $C$ representing the principal moments of inertia, $\omega_{1}$, $\omega_{2}$, $\omega_{3}$ being the $xyz$ components of the angular velocity $\boldsymbol\omega$, and $\Gamma_{1}$, $\Gamma_{2}$, $\Gamma_{3}$ are the components of the total moments of all the forces $\mathbf{\Gamma}$ when $\mathbf{\Gamma} = (\Gamma_{1}, \Gamma_{2}, \Gamma_{3})$.
\begin{equation}\label{eq:Gamma1}
	A\dot{\omega_{1}} - (B-C)\omega_{2}\omega_{3} = \Gamma_{1}
\end{equation}
\begin{equation}\label{eq:Gamma2}
	B\dot{\omega_{2}} - (C-A)\omega_{3}\omega_{1} = \Gamma_{2}
\end{equation}
\begin{equation}\label{eq:Gamma3}
	C\dot{\omega_{3}} - (A-B)\omega_{1}\omega_{2} = \Gamma_{3}
\end{equation}

These equations are not used in the application and would only be used if functionality were to be added which would allow the user to set a point on the object which is fixed so that it would pivot around it.