\subsection{Comparison:}
To see how well the application is at simulating general motion of rigid bodies, this section will look to compare the position the application has after a moment of time has occurred to the position calculated by hand at the same moment.
Figure \ref{fig:ScreenShotSingle} shows the cube after 0.526 seconds.
This time will be used when calculating position by hand using Equation \ref{eq:PositionGM} because in this scenario force is available meaning the acceleration can be worked out using Newton's Second Law $F=MA$.
\begin{equation}\label{eq:PositionGM}
\mathbf{p}=\frac{1}{2}\mathbf{A}t^{2}+{R}\tilde{\mathbf{p}_{0}}
\end{equation}
For the hand calculation the angular velocity will be represented by $\boldsymbol{\omega}$ and will be taken from the application.
The force direction vector will also be taken form the application and will be represented by $F$.
Also shown in \ref{var:Rotation Variables} is $\tilde{\mathbf{p}_{0}}$ and this represents the initial position vector.
Equations \ref{var:ftm} show three of the variables used $M$ beings the mass of the object, $f$ is the force applied to the object, and $t$ is the time since the initial force is applied. 
\begin{figure}[h!]
	\centering
	\includegraphics[width=\textwidth]{images/Screenshot2.PNG}
	\caption{Screenshot of Cube at $t = 0.526 s$}
	\label{fig:ScreenShotSingle}
\end{figure}
\begin{equation}\label{var:Rotation Variables}
	\boldsymbol{\omega} = 
	\begin{bmatrix}
		 2.44 \\
		-2.97 \\
		-1.36 
	\end{bmatrix}
	\qquad
	\mathbf{F} = f
	\begin{bmatrix}
		 0.6  \\
		 0.13 \\
		 0.79 
	\end{bmatrix}
	\qquad
	\tilde{\mathbf{p}_{0}} = 
	\begin{bmatrix}
		-2.5 \\
		 0 	 \\
		-2.5 
	\end{bmatrix}
\end{equation}
\begin{equation}\label{var:ftm}
	f = 5N
	\qquad
	t = 0.526s
	\qquad
	M = 1kg
\end{equation}
\begin{equation}\label{var:thetaomega}
	\omega = |\boldsymbol\omega| = 4.077
	\qquad
	\theta = \omega t = 2.144
\end{equation}
To start first acceleration will have to be calculated this will be represented by $\mathbf{A}$.
It will be calculated as mentioned before using Newton’s Second Law $F=MA$.
This means $\mathbf{A} = \mathbf{F}/M$ and results in Equation \ref{eq:Acceleration}.
\begin{equation}\label{eq:Acceleration}
	\mathbf{A} = \frac{f}{M}
	\begin{bmatrix}
		 0.6  \\
		 0.13 \\
		 0.79 
	\end{bmatrix} = 5
	\begin{bmatrix}
		 0.6  \\
		 0.13 \\
		 0.79 
	\end{bmatrix} = 
	\begin{bmatrix}
		 3  \\
		 0.65 \\
		 3.95 
	\end{bmatrix}
\end{equation}
The next step is to calculate the matrix $R$, which is a standard rotational matrix about any axis.
Equation \ref{ma:GeneralRotation} is the initial matrix that will be used. $C$ is shorthand for $(1-\cos\theta)$.
Below the rotational matrix are the values for $\alpha, \beta, \gamma$ in Equation \ref{var:alphabetagamma} and in Equation \ref{var:cs1-c} are the values of $\cos\theta, \sin\theta, (1-\cos\theta)$.
\begin{equation}\label{ma:GeneralRotation}
	R = 
	\begin{bmatrix}
		{\alpha}^{2}(C) + \cos\theta & 
		\alpha\beta(C) - \gamma\sin\theta & 
		\alpha\gamma(C) + \beta\sin\theta \\
		
		\alpha\beta(C) + \gamma\sin\theta & 
		{\beta}^{2}(C) + \cos\theta & 
		\beta\gamma(C) - \alpha\sin\theta \\
		
		\alpha\gamma(C) - \beta\sin\theta & 
		\beta\gamma(C) + \alpha\sin\theta & 
		{\gamma}^{2}(C) + \cos\theta \\
	\end{bmatrix}
\end{equation}
\begin{equation}\label{var:alphabetagamma}
	\alpha = \frac{\boldsymbol\omega_{x}}{\omega} = 0.5984
	\qquad
	\beta = \frac{\boldsymbol\omega_{y}}{\omega} = -0.7284
	\qquad
	\gamma = \frac{\boldsymbol\omega_{z}}{\omega} = -0.3336
\end{equation}
\begin{equation}\label{var:cs1-c}
	\cos\theta = -0.5421
	\qquad
	\sin\theta = 0.8403
	\qquad
	1-\cos\theta = 1.5421
\end{equation}
After substituting $\alpha, \beta, \gamma,\cos\theta, \sin\theta, $and $(1-\cos\theta)$ into the matrix the result can be seen in Matrix \ref{ma:Calculated Rotation}. 
\begin{equation}\label{ma:Calculated Rotation}
	R = 
		\begin{bmatrix}
		   0.0102 & 
		   0 & 
		   0.0721 \\
		  
		  -0.0016 & 
		   0.9999 & 
		  -0.0142 \\
		  
		  -0.0721 & 
		   0.0141 & 
		   0.9973 \\
	\end{bmatrix}
\end{equation}
\begin{figure}[h!]
	\centering
	\includegraphics[width=\textwidth]{images/Screenshot1.PNG}
	\caption{Screenshot of Cube at $t = 0.078 s$, $t = 0.296 s$, $t = 0.373 s$, $t = 0.498 s$}
	\label{fig:ScreenShotFour}
\end{figure}