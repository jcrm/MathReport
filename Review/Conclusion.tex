\subsection{Conclusion:}\label{sc:Con}
Looking at the accuracy of the application compared to the hand calculated results the Unity3D game engine might not have been the best way to create the simulation.
A better way could have been to create the application for scratch for a number of reasons.
The first being that it could have given the user better control over physics of the application for example allowing the user to manually set the angular velocity, or the time steps to take screenshots at.
Another reason would be because it would allow for better debugging because output information code be provided for every stage of linear and angular motion.

As well as possibly changing how the application was developed if the project was under taken again a few other improvements could have been made to the application.
A major one would be add the functionality to change the object the force is being applied to or by adding multiple objects.
Instead of just changing what object is being affected the user could be allowed to draw and create objects in real time and see what effects the force being applied has on it.
This would mean needing to work out the centre of mass of the object once it is created and the equations for working out the centre of mass from section \ref{sc:COM} would be helpful in doing so. 

Overall the project was fairly successful, there were places were improvements could be made such as in the accuracy of the final results.
However the application meet the specification of showcasing general motion of a rigid body when a force is applied at a particular point on the object. 
